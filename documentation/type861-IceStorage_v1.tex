% SPF Report:
\documentclass[english]{SPFReport}

%\usepackage{spfFigures}
\usepackage{longtable}
\usepackage{amsfonts}
\usepackage{color,xcolor}
\usepackage{listings}
\usepackage{footnote}
\usepackage{longtable}
\usepackage{siunitx}
\usepackage[sfdefault,condensed]{roboto}
\sisetup{detect-all=true,text-rm=\robotocondensed}

\renewcommand{\thefootnote}{\thempfootnote}

% Environment Variables:
\newcommand{\ud}{\mathrm{d}}
\newcommand{\HRule}{\rule{\linewidth}{0.5mm}}
%\renewcommand{\thefootnote}{\fnsymbol{footnote}} 	


\definecolor{dkgreen}{rgb}{0,0.6,0}
\definecolor{gray}{rgb}{0.5,0.5,0.5}
\definecolor{mauve}{rgb}{0.58,0,0.82}
 
%\lstset{ %
%  language=[90]Fortran,           % the language of the code
%  basicstyle=\footnotesize\ttfamily,       % the size of the fonts that are used for the code
%  numbers=left,                   % where to put the line-numbers
%  numberstyle=\tiny\color{gray},  % the style that is used for the line-numbers
%  stepnumber=1000,                   % the step between two line-numbers.
%  numbersep=1pt,                  % how far the line-numbers are from the code
%  backgroundcolor=\color{white},  % choose the background color. You must add \usepackage{color}
%  showspaces=false,               % show spaces adding particular underscores
%  showstringspaces=false,         % underline spaces within strings
%  showtabs=false,                 % show tabs within strings adding particular underscores
%  extendedchars=true,            % allow special characters
%  rulecolor=\color{black},        % if not set, the frame-color may be changed on line-breaks
%  tabsize=1,                      % sets default tabsize to 2 spaces
%  captionpos=b,                   % sets the caption-position to bottom
%  breaklines=true,                % sets automatic line breaking
%  breakatwhitespace=false,        % sets if automatic breaks should only happen at whitespace
%  keywordstyle=\color{blue},      % keyword style
%  commentstyle=\color{dkgreen},   % comment style
%  stringstyle=\color{mauve},      % string literal style
%  morekeywords={double,precision}            % if you want to add more keywords to the set
%}

\reportName{TRNSYS Type~861 }
\reportSubName{Ice Storage Model with ice-on-heat exchangers}

\reportDate{21$^{st}$ October of 2018} % or writte the date manually 
\author{Dr. Daniel Carbonell}
\address{Dani.Carbonell@spf.ch}

\abstract{ 
This TRNSYS Type simulates a ice storage with ice-on-coil, ice-on-plates and ice-on-capillary mat heat exchangers.
This storage typer does not allow two types in the same deck and direct ports (wihout heat exchangers) are not implemented.
This Type is compatible for TRNSYS 17 and 18
}

\begin{document} 


\section{Introduction}

The mathematical formulation of this model has been presented in \citet{carbonell_experimental_2018} for capillary mats and coils and in \citet{ICEEX_2017} for flat plate heat exchangers.

\section{Revison history}

  \begin{tabular}{|p{40mm}|p{15mm}|p{101mm}|}\hline    
    \textbf{Date} & \textbf{Version} & \textbf{Changes}  \\\hline
     Oct 31, 2018 & v1.0 & First version released \\\hline
  \end{tabular}

\section{Notes}

\begin{itemize}
\item The model does not allow to freeze completely one control volume of the storage tank. The energy balance is applied to the water of each control volume. Thus, if there is none or too little volume of water available, the model suffers from instability
\item For stability of the complete system it is recommended to bypass the ice storage once completely iced
\item The model has never been tested for other PCM, but probably with minor modifications it should work
\end{itemize}

\section{List of parameters}

%\begin{minipage}
\def\arraystretch{1.3}
\begin{tabular}{| l |  m{8cm} | l | l |}


\hline
\textbf{Nr.} & \textbf{Description} & \textbf{Name}& \textbf{Units} \\
\hline

 1 & unused &  & -  \\
  2 & Storage Volume & \si{V_{tes}} & \si{m^3}  \\
  3 & Storage height & \si{H_{tes}} & m  \\
  4 & Storage width & \si{W_{tes}} & m  \\
  5 & Storage geometry & \si{s_{type}} & - \\    
     &   0: box 1:cylinder (\si{W_{tes}} not used) &&\\
  6 & Distance between pipes if \si{hx_{type}} = 1 & \si{x_1} & m \\
     & Distance beween hx if \si{hx_{type}} = 2 & & \\
  7 & Distance between hx if \si{hx_{type}} = 1 & \si{x_2} & -  \\
  8 & Effective thermal conductivity of store & \si{k_{wat}} & \si{W/(mK)}  \\
  9 & Density of storage fluid (water) & \si{$\rho_{wat}$} & \si{kg/m^3}  \\
  10 & Specific heat storage fluid (water) & \si{c_{p,wat}} & J/(kgK) \\
%  11 & Density ice & \si{$\rho_{ice}$} & \si{kg/m^3}  \\
11 &not used &  &  \\
  12 & Thermal conductivity of ice & \si{k_{ice}} & \si{W/(mk)}  \\
  13 & Water$\leftrightarrow$ice enthalpy & \si{$\Delta$H_{ice}} &        \si{kJ/kg}  \\
  14 & Subcooling temp & \si{T_{subcool}} & \si{\degreeCelsius} \\
  15 & Freezing temperature & \si{T_{freeze}} &\si{\degreeCelsius} \\
  16 & Initial amount of ice in store & \si{kg_{ice}} & kg \\
  17 & Critical film melting thickness & \si{$\delta$_{melt}} & m \\
  18 & Maximum storage ice fraction & \si{$\zeta$_{tes}} & - \\
  19 & Parameter check control &&\\
       &  1 : interrups simulation if parameters out of range && \\
20 & unused &&\\
 21 & hx height respect to the total ([0-1])& \si{hx_{height}} & m\\ 
 22 & maximum ice floating ratio in layers were ice is produced (only relevant when deicing is used) & &\\
 23 & Use of \si{T_{wall}} from previous time step (improves convergence) & &m\\
 24 & Number fo used hx (only used for \si{hx_{type}}=1 (capillary mats) &&\\
 25 & use constabnt physical properties in the water from the storage && \\
 26 & use corrugated configuration for flat plates (only used when \si{hx_{type}}=2 &&\\
 \multicolumn{4}{c}{\small -- continued on next page --} \\
\end{tabular}

%\end{minipage}

\begin{tabular}{| l |  m{8cm} | l | l |}
\hline
%%%%%%%%%HEAT EXCHANGER
     & \textbf{Heat Exchanger i:} & & \\
  \small{ 27 + 19\footnote{i=1,4 which is the maximum number of heat exchangers}(i-1) } &Hx geometry & \si{hx_{type}}& \\  
   &0=flat plates, 1=capilary mats,2=coils &&\\
  \small{ 28 + 19(i-1)}& Number of hx & \si{N_{hx}} & \\  
  \small{ 29 + 19(i-1)}& \si{hx_{type}}=0  inner tube diameter & \si{d_{in,hx}} & \\  
                           & \si{hx_{type}}=1 height flat plate & \si{H_{hx}}& \\  
  \small{ 30 + 19(i-1)}& \si{hx_{type}}=0 outer tube diameter &\si{d_{out,hx}}& \\  
                           & \si{hx_{type}}=1  width of flat plate &\si{W_{hx}}& \\  
  \small{ 31 + 19(i-1)}& Lenght of hx  & \si{L_{hx}} & \\  
  \small{ 32 + 19(i-1)}& \si{hx_{type}}=0  additional heat capacity hx & & \si{J/m^3} \\  
                           & \si{hx_{type}}=1  wall thickness flat plate &\si{dx_{wall,hx}} & m \\  
  \small{33+19(i-1)}& Order of hx from 1 to 4. Used only if seriesMode is active  &  & \\  
  \small{34+19(i-1)}& Hx thermal conductivity  & \si{$\lambda$_{hx,i}} & \si{W/(mK)}\\
  \small{35+19(i-1)}& Heihgt of inlet on the hx & [0-1]?? & m \\
   \small{36+19(i-1)}& Heihgt of outlet on the hx & [0-1]?? & m \\
 
   \small{37+19(i-1)}& Fluid thermal conductivity & \si{$\lambda$_{p,i}} & W/(mK) \\
   \small{38+19(i-1)}& Fluid specific heat & \si{c_{p,i}} & J/(kgK) \\
   
   \small{39+19(i-1)}  & Glycol concentration &  &  \%\\
    \small{40+19(i-1)}  & C Factor for  Nusselt correlation (heating)  &  \si{C_{hx,i}} & \\
    \small{41+19(i-1)}  & n Factor  for  Nusselt correlation (heating) &  \si{n_{hx,i}} & \\
     \small{42+19(i-1)}  & C Factor  for  Nusselt correlation (cooling)  &  \si{C_{hx,i}} & \\
     \small{43+19(i-1)}  & n Factor  for Nusselt correlation (cooling) &  \si{n_{hx,i}} & \\
\small{44+19(i-1)} & Enhanced Nusselt number for laminar flow (\si{Nu = Nu$\cdot$ Nu_{en}}) & \si{Nu_{en}} & - \\
      & \textbf{End of Heat Exchanger } & & \\
      \hline                                            
104 & U lower lateral side   & \si{U_{low,lat}} & \si{W/(m^2K)}\\
105 &  U upper lateral side   & \si{U_{up,lat}} &  \si{W/(m^2K)}\\
106 & U bottom side   & \si{U_{bot}} &  \si{W/(m^2K)}\\
107 & U upper side   & \si{U_{top}} &  \si{W/(m^2K)}\\
108 & $1^{th}$ sensor height position  & \si{y_{s1}} & m\\
109 & $2^{th}$ sensor height position  & \si{y_{s2}} & m\\
110 & $3^{th}$ sensor height position  & \si{y_{s3}} & m\\
111 & $4^{th}$ sensor height position  & \si{y_{s4}} & m\\
112 & $5^{th}$ sensor height position  & \si{y_{s5}} & m\\
\hline

\end{tabular}

%\vdots   & for j=1 j $\leq$ $nCv$ j=j+1  & \vdots & \vdots \\



\section{List of inputs}


\begin{tabular}{| l |  m{8cm} | l | l |}
\hline
\hline
\textbf{Nr.} & \textbf{Description}& \textbf{Name} & \textbf{Units} \\
\small{1+3(i-1)\footnote{i=1 to 4 which is the maximum number of heat exchangers}}  & Inlet fluid temperature of the hx $i$&  \si{T_{in,hx,i}} & \si{\degreeCelsius}  \\
\small{2+3(i-1)}  & Inlet mass flow rate of the hx $i$&  \si{$\dot m$_{in,hx,i}} & \si{kg/h} \\
\small{3+3(i-1)}  & Reverted temperature of the hx $i$ &  \si{T_{rev,hx,i}}  & \si{\degreeCelsius} \\
& Temperature used when \si{$\dot m$_{in,hx,i}<0} &  &\\
\vdots  &for i=1 i $\leq$4 i=i+1   & \vdots & \vdots\\
13  & 0 : mechanical de-ice off ; 1 : on  &  &   \\
14+(n-1)  & Surrounding temperature around the TES (for heat loss calculation) & \si{T_{amb,n}} & \si{\degreeCelsius} \\
\vdots   &  for n=1 n $\leq$nCv n=n+1 & \vdots & \vdots \\
14+(nCv)  & Surrounding temperature below TES (for heat loss calculation) &  & \si{\degreeCelsius} \\
14+(nCv+1)  & Surrounding temperature above TES (for heat loss calculation) &  & \si{\degreeCelsius} \\
\hline
\hline
\end{tabular}


\section{List of derivatives}

The definition of derivatives is used to calculate the number of control volumes in the storage $nCv$

\begin{tabular}{| l |  m{8cm} | l | l |}
\hline
\hline
\textbf{Nr.} & \textbf{Description}& \textbf{Name} & \textbf{Units} \\
\small{n\footnote{n=1 to nCv which is the defined number of Cv}}  & Initial storage temperature for each Cv n&  \si{T_{ini,n}} & \si{\degreeCelsius} \\
\vdots  &for n=1 n $\leq$ nCv n=n+1   & \vdots & \vdots \\
\hline
\end{tabular}

\section{List of outputs}

\begin{longtable}{| l |  m{8cm} | l | l |}
\hline
\textbf{Nr.} & \textbf{Description}  & \textbf{Name}& \textbf{Units} \\
\hline
1  & Average temperature of the store & \si{$\dot T$_{s,av}} &\si{\degreeCelsius}   \\
2  & Total heat provided by the Hx to the storage tank & \si{$\dot Q_{hx}$}& W \\
3  & Total heat accumulated  of the storage tank& \si{$\dot Q_{acum}$} & W \\
4 & Total heat losses of the storage tank& \si{$\dot Q_{loss}$} & W \\
5 & Total heat used to melt the ice in the storage tank& \si{$\dot Q_{melt}$} & W \\
6 & Total heat used to form the ice & \si{$\dot Q_{ice}$} & W \\
7 & Total imbalance heat & \si{$\dot Q_{imb}$} & W \\
8 & Mass of floating ice & \si{$\dot M_{ice,f}$} & kg \\
9 & Total ice thickness in the heat exchangers & \si{ds_{ice}} & m \\
10 & Total mass of ice (floating + hx,ice) & \si{$\dot M_{ice,T}$} & kg \\
11 & 1: Upper part of the storage is full of ice (no ice can be released); 0: not full &  &\\
\small{12 + 7(i-1)\footnote{i=1 to 4 which is the maximum number of heat exchangers}} & Inlet temperature of heat exchanger $i$ & \si{T_{in,hx,i}} &\si{\degreeCelsius} \\
\small{13 + 7(i-1)} & Outlet temperature of heat exchanger $i$ & \si{T_{out,hx,i}} &\si{\degreeCelsius} \\
\small{14 + 7(i-1) }& Wall temperature of heat exchanger $i$ & \si{T_{wall,hx,i}} &\si{\degreeCelsius}\\
\small{15 + 7(i-1)} & Power provided for all parallel heat exchangers $i$ & \si{$\dot Q_{hx,i}$} & W \\
\small{16 + 7(i-1)} &  Ice thickness in the heat exchangers $i$ & \si{ds_{form,i}} & m \\
\small{17 + 7(i-1)} &  Ice melted in the heat exchangers $i$ & \si{ds_{melt,i}} & m \\
\small{18 + 7(i-1)} & Total heat transfer coefficient for all parallel heat exchangers $i$ & \si{UA_{hx,i}} & W/K\\
\vdots   &  for i=1 i $\leq$4 i=i+1 & \vdots & \vdots\\
40 & 1: Heat exchangers are full of ice (collapse of ice storage); 0: not full &  & \\
41 +(n-1) & Temperatrue of sensor n &&\si{\degreeCelsius}\\
46 +(n-1) & Temperature of the storage for the Cv $n$ & \si{T_{n}} & \si{\degreeCelsius}   \\
\vdots   &  for n=1 n $\leq$ nCv n=n+1 & \vdots & \vdots \\
56 & Losses at the bottom & \si{q_{loss,bottom}} & W   \\
57 +(n-1) & Losses for the Cv $n$ & \si{q_{loss,n}} & W   \\
\vdots   &  for n=1 n $\leq$ nCv n=n+1 & \vdots & \vdots \\
67 & Losses at the top & \si{q_{loss,top}} & W  \\
68 & Temperature of the storageat the bottom & \si{T_{bottom}} & \si{\degreeCelsius}   \\
69 & Temperature of the storageat the top & \si{T_{top}} & \si{\degreeCelsius}   \\
70 & Heat transfer coefficient at the bottom & \si{U_{bottom}} & \si{W/(m^2K)}\\
71 & Average heat transfer coefficient at the side of the storage & \si{U_{bottom}} &  \si{W/(m^2K)} \\
72 & Heat transfer coefficient at the top & \si{U_{zop}} & \si{W/(m^2K)} \\
\hline
 %\multicolumn{4}{c}{\small -- continued on next page --} \\
%\hline
\end{longtable}




% Bibliography
\clearpage
\bibliographystyle{apa}
\bibliography{local}



\end{document}